\begin{document}
\begin{center}
    Kas's Network Model notes of DOOM
\end{center}

\begin{itemize}
    \item What are the 7 layers of the OSI model?\\
    Application, presentation, session, transport, network, data link, and physical.
    Layers 7->1\\
    \item True or False layers are peer-to-peer.\\
        True\\
    \item What does the physical layer do?\\
    The physical layer moves bit sequences over a physical link.

    \item Why is the physical layer important?\\
    It defines how data is physically moved from point to point. Affecting protocols,
    design and complexity of upper layers, and it may reduce the importance of or 
    completely remove some layers.

    \item What is the PDU in the physical layer?\\
    Trick question. There is none.

    \item List 4 things defined in the physical layer.\\
    Physical characteristics of EIA 422/485 balanced mode interfaces and medium.\\
    Bit representation.\\

    Data rate\\

    Bit sync: Sender and reciever clock sync at the same rate\\

    Line configuration: Point-to-point or Multipoint.\\

    Physical topology: Mesh, ring, bus, or hybrid.\\

    Transfer mode: Simples, F/D, and H/D.\\

    Physical media: TP, fiber optics, wireless,...\\

    \item What are the two parts of the Data link layer?\\
    Logic Link Control (LLC) and Multi Access Control (Mac)
    forming the PDU and the address respectively.\\


    \item What is the PDU of the data link layer?\\
    It is a MAC sublayer frame with a header and a trailer.

    \item What is the address of the data link layer.?\\
    Mac address or MAC sublayer and the server access point (SAP) for the LLC sublayer.


    \item What are two things the LLC does?\\
    Framing/Deframing\\

    Physical Addressing: sender/reciever address in the same header.\\

    Flow Control: to prevent a faster sender from flooding a slower reciever.\\

    Error Control: Increase physical reliablity by adding a mechanism to detect 
    ReTx damages and lost frames. (trailer)//

    \item Why do we want error control in the Data link layer?\\
    Because at layer 4 it is more costly to correct errors rather than fixing it through an 
    entire packet. Improving overall efficency.

    \item What does the MAC do?\\
    Control the access between connected all devices and manage bridging by ensuring that 
    different protocol syntax and semantics are resolved.\\

    \item what is bridging?\\
    Using moving data through multiple LANs.\\

    \item What is the third layer?
    The networking layer.

    \item what is a PDU?\\
    Protocol data unit.

    \item What is the PDU and address of the Networking layer?\\
    PDU:packet\\
    Address: Logical address (IP)\\

    \item What are the functions of the Networking layer? Explain one\\
    Routing and Internetworking.\\
    Routing is determing what the next path in the network to take in order to 
    determine the most efficent and/or secure path.\\
    Internetworking is resolving any Network protocol conflicts while moving 
    packets over a subnet of different protocols.\\

    \item Why do we need Logical addressing.
    Physical addresses are not enough to put in just the Data Link we need it in the header 
    of the sender and reciever.\\

    \item What is the fourth layer?\\
    The Transport Layer.\\

    \item What is the PDU and Addressing in the Transport Layer?\\
    PDU: Segment\\
    Address: Service Access Point (SAP)\\

    \item Why is the transport layer important?\\
    It provides abstraction and optimization of the complex detials of the subnet. Allowing for quality 
    of service over the Network Layer protcol.\\

    \item What are the responsibilites and provide an explanation of each.\\
    Muxing/Demuxing: Converting data into Segments between host processes, at different 
    machines, over the subnet trying for the most optimal use of the so it may Mux or Demux 
    of segments over the Network layer?\\
    Service Access Point Addressing: Network Layer logical address from source host to destination host
    thus we need another address mechanism for SAP addressing within the same host system for user's 
    process and message delievery.\\
    Segmentation and deassembly: Converting segments into packets or vice versa. Each seperation 
    is given a sequence for recombining later at the reciever.\\
    Connection Control: How will data arrive Connection or connectionless.\\

    Flow Control: Similar to Message Link but at the message level. What process is given priority.\\

    Error Control: Catches errors in messages.\\


    \item what is the difference between Muxing and DeMuxing?\\
    Muxing will combine smaller packets into a much larger one for efficency.\\
    DeMuxig will unwrap a larger piece of data into much smaller segments to 
    send of the network for speed.\\

    \item Why is error control in the transportation layer slower?\\
    When a singular error is caught it will request that the sender recieve the entire 
    message a second time. Where the Data link layer will only request a singular frame.


    \item What are the 3 physical layers?\\
    Layers 1-3 but the Physical, Datalink, and Network layers.\\

    \item Where do the other layers occur?\\
    On the sender/recievers machine.

    \item what is the 5th layer?\\
    Session Layer

    \item What does the session layer do?\\
    Dialog controls: Half-Duplex (H/D) or Full-Duplex (F/D)\\
    Synchronization: Checkpoints are added to the data streams for dividing into units
    of indepdent ACK. Communcation robostness in case of crashes.\\


    \item What is the 6th layer?\\
    The presentation layer.\\

    \item What does the presentation layer do?\\
    Translations: ASCII, EBCDIC, abstract syntax notation...\\
    Encryption: Securing information for text privacy.\\
    Compression: For efficent utlization of bandwidth.\\

    \item What is the 7th layer of Networks?\\
    The Application layer?\\

    \item What are two purposes of the Application layer?\\
    Virtual Terminal: Putty to allow remote logins (emulations).\\
    File Transfer\\
    Mail Service\\
    Directory Service: HTTP for example\\
\end{itemize}
\end{document}